\documentclass[11 pt]{article}
\usepackage[T1]{fontenc}
\usepackage{bbold}
\usepackage{algorithm}
\usepackage{amsmath, amssymb}
\usepackage{dsfont}
\usepackage{algpseudocodex}
\begin{document}
\newtheorem{theorem}{Theorem}
\title{rapport de Stage}
\date{2023\\ Juillet}
\author{Romain Gambardella}
\maketitle
\section{Réalisations}

J'ai commencé par créer une base de donnée sur le jeu du Futoshiki et faire fonctionner le Futoshiki avec l'EPLL ( avant le début du stage ).

En fait, après le début du stage je me suis rendu compte qu'il y avait plusieurs comportements bizarres : 
\begin{enumerate}
	\item un gradient était toujours nul lorsqu'il y avait une inégalité
	\item des termes négatifs apparaissaient dans les matrices de coûts des inégalitées, ce qui posait de gros problèmes à toulbar pour la résolution
\end{enumerate}
Le point 1) a été expliquée par l'expression du gradient : 
pour $i > j $ \\
$\frac{\partial l_{PL}(n, \theta )}{\partial \theta_{ij}[v_i,v_j]} = 2 \times \mathbb{1}(y_i = v_i, y_j = v_j) - \mathbb{P}(y_i=v_i) \mathbb{1}(y_j = v_j ) -   \mathbb{P}(y_j=v_j) \mathbb{1}(y_i = v_i ) $
Dans le cas où les cases i et j ne prennent jamais les valeurs $v_i$ et $v_j$, le gradient est toujours nul.

J'ai essayé plusieurs méthodes pour compenser ce problème : 
\begin{enumerate}
	\item dans le cas du Sudoku, ajouter des termes unaires ne fonctionne pas car les termes unaires ne fonctionnent pas bien, à cause de la superposition des contraintes
	\item rajouter du "bruit" casse complètement la PLL.
	\item implementer une PLL stochastique à l'ordre 2 ( et à l'odre n ) vectorisée ( fonctionne ) 
\end{enumerate}

J'ai généralisé le code de Marianne pour que je puisse l'appliquer aux prochains problèmes plus facilement ( par exemple le grounding ) -> le code marche maintenant sur une version plus complexe du visual sudoku, sans les indices.
J'ai vectorisé le code de gestion des plus proches voisins.





\section{PLL stochastique}
\begin{algorithm}
\caption{PLL stochastique}\label{alg:pll}
\begin{algorithmic}
\Require $\theta '$ le modèle actuel 
\Require $neighbours$ une fonction stochastique qui à x associe un ensemble d'ensemble de voisins ( dans le cas de la PLL à l'ordre 1, neighbours est déterministe et correspond l'ensemble des  $V_i$ tels que $V_i = \{y \text{ tel que $x$ et $y$ coincident partout sauf à l'indice $i$}\}$
\Function{PLL}{x}
\State $pll \gets 1$
\State $N \gets neighbours(x) $
\ForAll{$N_i \in N$}
\State $proba \gets \mathbb{P}_{\theta '} (X=x \mid X \in N_i)$  
\EndFor
\Return $pll$
\EndFunction
\While 1
\State $x \gets$ une réalisation de $\theta$
\State $pll \gets PLL(x)$
\State $\theta ' \gets update(\theta ', -log(PLL))$
\EndWhile
\end{algorithmic}
\end{algorithm}
\begin{algorithm}
\caption{Fonction neighbours pour la PLL d'ordre 2 stochastique}\label{alg:neig2}
\begin{algorithmic}
	\Require $N_{masks}$ le nombre de couple d'indices à générer ( $Card(N) = N_{masks} $)
	\Require $N_{2-uplets}$ le nombre de 2-uplets à étudier sur ces indices\\ ( $ N_{2-uplets} = Card(N_i) $)
\Function{neighbours}{x}
\State $masks \gets  \{ N_{masks} \text{ 2-tuples d'indices aléatoires} \}$
\State $N \gets \{\}$
\ForAll{ $n_{mask} \in 0...N_{masks}$}
\State $G \gets \{ y \in \Omega \text{ tels que } y \neq x, y_i = x_i \forall i \notin masks[n_{mask}] \}$
\State $N_{n_{mask}} \gets \{choisir N_{2-uplets}\; y\; dans \; G\} \cup \{x\}$
\State ajouter $N_{n_{mask}}$ à N
\EndFor
\Return $N$
\EndFunction
\end{algorithmic}
\end{algorithm}

\newpage

\subsection{Cohérence de la PLL d'odre 2 stochastique}
Let $\theta'$ be the computed model and $\theta$ the observed model. Let $\Omega_x$ denote all the possible observations.

\quad

Let  $ PLL(\theta')_m)$ be the value of the PLL averaged over m random samples ( observations ) $X_1,...,X_m$ i.i.d that follows the distribution of $\theta$, that is : 

\quad

$PLL(\theta')_m = \sum_{n=0}^{m} PLL(X_n) $

\quad

We denote the random neighbourhood used to compute the stochastic $PLL(X_i)$ by $N(X_i)$.

\begin{theorem}
Let N be the random fonction used to compute the stochastic PLL.\\
Suppose that $\mathbb{P}_{}(N(X) = n  \mid X = x) = cte(n), \; \forall  \; n \in \mathcal{P}(\Omega_x)  \; and \; \forall x \in n$
Then : 
\begin{enumerate}
	\item $ PLL(\theta')_m $ converges when m tends to $\infty$ to a number, that we call $PLL(\theta')$.
	\item PLL($\theta'$) is minimum at $\theta' = \theta$.
\end{enumerate}

\end{theorem}

\quad 

We first give another expression for $PLL(\theta')_m$:

\quad


$PLL(\theta')_m = \frac{1}{m} \sum_{n=1}^{m} PLL(X_n)\\
		= \frac{1}{m} \sum_{n=1}^{\infty} \sum_{n \in N(X_i)}  \mathbb{P}_{\theta '} (Y=X_i \mid Y \in n) \\
		= \frac{1}{m} \sum_{n=1}^{\infty} \sum_{x \in \Omega_x} \sum_{n \in \mathcal{P}(\Omega_x)} \mathbb{1}(X_n = x, n = N(X_n) ) \mathbb{P}_{\theta '} (Y=x \mid Y \in n)
		$

\quad

Reordering the sums, we get: \\
$PLL(\theta')_m \\
		= \frac{1}{m} \sum_{n \in \mathcal{P}(\Omega_x)} \sum_{x \in \Omega_x} \sum_{n=1}^{\infty} \mathbb{1}(X_n = x, n = N(X_n) ) \mathbb{P}_{\theta '} (Y=x \mid Y \in n)
	\\	\to \sum_{n \in \mathcal{P}(\Omega_x)} \sum_{x \in \Omega_x} \mathbb{P}_d(X = x, n = N(X) ) \mathbb{P}_{\theta '} (Y=x \mid Y \in n)
		$
		
		by the law of large numbers, where $\mathbb{P}_d$ is the joint probability distribution of X, an input sample, and N(X).

		We have also : $\mathbb{P}_d(X = x, N(X) = n) = \mathbb{P}_d(X = x) \mathbb{P}_d(N(X) = n \mid X = x)  \\
		= \mathbb{P}_d(X \in n)  \mathbb{P}_d(X = x \mid X \in n)  \mathbb{P}_d(N(X) = n \mid X = x)  \\
		=  \mathbb{P}_{\theta}(X = x \mid X \in n) \times K(n)  \\
		$

		where $K(n)$ does not depend on X by assumption.

		We hence get : \\
		$ PLL(\theta')  =  \\
		\sum_{n \in \mathcal{P}(\Omega_x)} K(n) \sum_{x \in \Omega_x} \mathbb{P}_{\theta}(X = x \mid X \in n ) log (\mathbb{P}_{\theta '} (Y=x \mid Y \in n))
$

Which is minimum for $\theta' = \theta$


We immediatly deduce from this that the stochastic PLL of order 2 previously mentionned is indeed minimum at $\theta' = \theta$
\end{document}
